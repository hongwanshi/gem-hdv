% \VignetteIndexEntry{gdxrrw Tutorial}
% \VignetteDepends{gdxrrw}
% \VignettePackage{gdxrrw}
\documentclass[11pt]{article}
\usepackage{amsmath}
\usepackage[pdftex]{graphicx}
\usepackage{color}
% \usepackage{xspace}
% \usepackage{fancyvrb}
% \usepackage{fancyhdr}
\usepackage[colorlinks=true,
            linkcolor=blue,
            citecolor=blue,
            urlcolor=blue]{hyperref}
\usepackage{lscape}
\usepackage{Sweave}

\newcommand{\rrw}{\texttt{gdxrrw}}

\title{A Tutorial for the {\tt gdxrrw} Package}
\author{Steve Dirkse \\ GAMS Development Corporation}

\begin{document}

\maketitle

\thispagestyle{empty}
\section{Introduction}

In this tutorial introduction to \rrw{} we carry out a small modeling
exercise.  We consider a common use case in which one starts with a
self-contained, fully-functional GAMS model that reads from and writes
to GDX.  The GAMS source for the model is not R-specific; it works
equally well with or independently of R.  Without changing the model
source, we can control the model behavior by passing it flags that
control the GDX data input and output used.

It's worth noting that this tutorial assumes the user is familar with
GAMS and to a lesser extent with R.

\section{First steps}

Here we execute the basic steps: transferring data between GAMS and R,
and running GAMS.

Before you can use any R package, you must load
it.  The \rrw{} package also depends on the GDX shared libraries found
in the GAMS system directory.  The \texttt{igdx} command is useful for
setting and querying this linkage between \rrw{} and GAMS.  For
example, the usage below checks for the GAMS system directory in the
environment variable \texttt{R\_GAMS\_SYSDIR}.
<<loadPackage>>=
library(gdxrrw)
igdx('')
if (! igdx(silent=T)) stop ('Could not load GDX API')
@

Our starting point is a basic transport model where all of the data
inputs and outputs are done via GDX.  Running this model is easy once
we figure out how to get it from the \texttt{doc} subdirectory of the
package:
<<genData>>=
gms <- system.file('doc','transport.gms',package='gdxrrw', mustWork=T)
isWindows <- ("mingw32" == R.Version()$os)
if (isWindows) gms <- gsub("/","\\",gms,fixed=TRUE)
ingdx <- system.file('doc','inputs.gdx',package='gdxrrw', mustWork=T)
if (isWindows) ingdx <- gsub("/","\\",ingdx,fixed=TRUE)
rc <- gams(paste0(gms, " --INPUT=", ingdx))
if (0 != rc) stop ('GAMS run failed: rc = ',rc)
@

For reference, we have included the GDX file \texttt{outputs.gdx}
produced by the run above in the \texttt{doc} directory.
<<inCase>>=
if (! file.exists('outputs.gdx')) {
  ogdx <- system.file('doc','outputs.gdx',package='gdxrrw', mustWork=T)
  file.copy(ogdx,'outputs.gdx')
}
@

The model \texttt{transport} dumps all of its data (including its
inputs) to GDX before it quits.  There are 5 inputs: 2 sets and 3
parameters.
<<origData>>=
outgdx <- 'outputs.gdx'
if (! file.exists(outgdx)) stop (paste('File not found:',outgdx))
I <- rgdx.set(outgdx,'I')
J <- rgdx.set(outgdx,'J')
a <- rgdx.param(outgdx,'a')
b <- rgdx.param(outgdx,'b')
c <- rgdx.param(outgdx,'c')
@

As an exercise, we first generate a GDX file whose data is identical
to the original inputs, and verify that the solution with this data is unchanged:
<<identData>>=
wgdx.lst('intest',list(I,J,a,b,c))
rc <- system2('gdxdiff', paste('intest.gdx', ingdx), stdout=F)
if (0 != rc) stop ('gdxdiff says intest.gdx and inputs.gdx differ: rc = ',rc)
rc <- gams(paste(gms, '--INPUT intest.gdx --OUTPUT outtest.gdx'))
if (0 != rc) stop ('gams failed: rc = ',rc)
# system2('gams', 'transport.gms --INPUT intest.gdx --OUTPUT outtest.gdx')
zlst <- rgdx('outputs.gdx',list(name='z'))
z <- zlst$val
# we don't need to use the intermediate zlst variable to get the val
zz <- rgdx('outtest.gdx',list(name='z'))$val
if (0.0 != round(z-zz,6)) stop (paste("different objectives!! ", z, zz))
@

If we double all transportation costs, we can expect to double the objective.
<<doubleData>>=
c2 <- c
c2[[3]] <- c[[3]] * 2
wgdx.lst('in2',list(I,J,a,b,c2))
gams(paste(gms,'--INPUT in2.gdx --OUTPUT out2.gdx'))
z2 <- rgdx('out2.gdx',list(name='z'))$val
print(paste('original=', z,'  double=',z2))
@

\section{Using New Data}

We can define a new problem by changing the sets I and J.
For example, we can use the state data in R to create a set of source
and destination nodes.  The state data includes populations that we
can use to scale demands, and we can base transportation costs on the
lat/long data for the state centers.
<<statesData>>=
data(state)
src <- c('California','Washington','New York','Maryland')
dst <- setdiff(state.name,src)
supTotal <- 1001
demTotal <- 1000
srcPop <- state.x77[src,'Population']
srcPopTot <- sum(srcPop)
dstPop <- state.x77[dst,'Population']
dstPopTot <- sum(dstPop)
sup <- (srcPop / srcPopTot) * supTotal
dem <- (dstPop / dstPopTot) * demTotal
x <- state.center$x
names(x) <- state.name
y <- state.center$x
names(y) <- state.name
cost <- matrix(0,nrow=length(src),ncol=length(dst),dimnames=list(src,dst))
for (s in src) {
  for (d in dst) {
    cost[s,d] <- sqrt((x[s]-x[d])^2 + (y[s]-y[d])^2)
  }
}
@

Now that we've created the raw data for our transportation problem, we
need to put it in proper form for writing out the GDX.  In this example,
we use a list for each symbol to write, although we could use data
frames too.
<<statesGDX>>=
ilst <- list(name='I',uels=list(src),ts='supply states')
jlst <- list(name='J',uels=list(dst),ts='demand states')
suplst <- list(name='a',val=as.array(sup),uels=list(src),
               dim=1,form='full',type='parameter',ts='supply limits')
demlst <- list(name='b',val=as.array(dem),uels=list(dst),
               dim=1,form='full',type='parameter',ts='demand quantities')
clst <- list(name='c',val=cost,uels=list(src,dst),
             dim=2,form='full',type='parameter',
             ts='transportation costs')
wgdx.lst('inStates',list(ilst,jlst,suplst,demlst,clst))
@

Once the data is written to GDX we can call gams, as before.  A model
status of 1 (Optimal) indicates an optimal solution was found.
<<statesSolve>>=
gams(paste(gms,'--INPUT inStates.gdx --OUTPUT outStates.gdx'))
ms <- rgdx.scalar('outStates.gdx','modelStat')
print(paste('Model status:',ms))
@

\section{Abnormal Results}

Things often fail to go as planned.  In many cases, this is indicated
by a nonzero return code from a GAMS run.  For example, we could point
to a non-existent GDX input:
<<missingGDX>>=
rc <- gams('transport --INPUT notHere')
if (0 == rc) print ("SHOULD NOT GET HERE") else print ("abnormal gams return as expected")
@

Another case to consider is an infeasible model.  This is not an
error: infeasible models are sometimes expected.  The model status
(one of many values available after a solve) can indicate an infeasible
model, among other things.  If we reduce the supply available from
Seattle, we make the model infeasible.
<<infeas>>=
a2 <- a
rows <- a2$i == 'seattle'
a2[rows,2] <- a2[rows,2] - 100
wgdx.lst('inInf',list(I,J,a2,b,c))
gams(paste(gms,'--INPUT inInf.gdx --OUTPUT outInf.gdx'))
ms <- rgdx.scalar('outInf.gdx','modelStat')
if (4 == ms) print ("Reduced supply makes model infeasible")
@

\end{document}